\documentclass{article}

% Language setting
% Replace `english' with e.g. `spanish' to change the document language
\usepackage[english]{babel}

% Set page size and margins
% Replace `letterpaper' with `a4paper' for UK/EU standard size
\usepackage[letterpaper,top=2cm,bottom=2cm,left=3cm,right=3cm,marginparwidth=1.75cm]{geometry}

% Useful packages
\usepackage{amsmath}
\usepackage{graphicx}
\usepackage[colorlinks=true, allcolors=blue]{hyperref}

% quotes
\usepackage{dirtytalk}

% code blocks
\usepackage[outputdir=build, cachedir=build/_minted-notes]{minted}

\title{Lectures Notes on Cloud Computing}
\author{Camilo de Lellis}

\begin{document}
\maketitle

\tableofcontents

\section{Aula - 09/09/2025}
Conteúdos ministrados: Aula 00 - Apresentação da disciplina; - Aula 01 - Histórico e Contextualização dos sistemas de computação em Nuvem.

\subsection{Ementa da Disciplina}
\textbf{Curso:} Curso Superior de Tecnologia em Sistemas para Internet
\textbf{Disciplina:} Desenvolvimento Web para a Nuvem \textbf{Carga-Horária:} 60h (80h/a)
\textbf{Pré-Requisito(s):} Desenvolvimento Web Back-end \textbf{Número de créditos:} 4
\begin{center}
EMENTA
\end{center}
Conhecer o estado da arte sobre desenvolvimento web para a nuvem.
\begin{center}
PROGRAMA
\end{center}
\begin{center}
Objetivos
\end{center}
\begin{itemize}
      \item Aprender sobre os modelos de computação em nuvem;
      \item Conhecer os desafios do ambiente em nuvem;
      \item Conhecer cenários de utilização em ambientes em nuvem;
      \item Conhecer e desenvolver softwares como serviço.
\end{itemize}
\begin{center}
Bases Científico-Tecnológicas (Conteúdos)
\end{center}
\begin{itemize}
      \item 1. Princípios
      \begin{itemize}
           \item 1.1. Histórico e Contextualização dos sistemas de computação em Nuvem;
           \item 1.2. Introdução aos tipos de nuvens;
           \item 1.3. Benefícios, desafios e riscos das plataformas de serviços;
           \item 1.4. Cenários de Utilização;
           \item 1.5. Modelos de negócios aplicáveis às nuvens.
      \end{itemize}
      \item 2. Modelos de Computação em Nuvem
      \begin{itemize}
            \item 2.1. Software como serviço (SaaS);
            \item 2.2. Infraestrutura como serviço (Iaas);
            \item 2.3. Plataforma como serviço (PaaS);
            \item 2.4. Middlewares para computação em nuvem.
      \end{itemize}
      \item 3. Configuração
      \begin{itemize}
            \item 3.1. Administração e Regras;
            \item 3.2. Escalonamento;
            \item 3.3. Balanceamento de recursos em computação nas nuvens.
      \end{itemize}
      \item 4. Desafios de Programação para Computação em Nuvem
      \begin{itemize}
            \item 4.1. Segurança;
            \item 4.2. Privacidade;
            \item 4.3. Legado
            \begin{itemize}
                  \item 4.3.1. Migração de sistemas para nuvem.
            \end{itemize}
      \end{itemize}
      \item 5. Gerenciamento de Dados e Otimização
      \begin{itemize}
            \item 5.1. Gerenciamento de dados e desafios de manutenção nos sistemas de computação nas nuvens;
            \item 5.2. Visão geral de técnicas de otimização incluindo o gerenciamento de consumo de energia elétrica.
      \end{itemize}
      \item 6. Migração e Transformação de Servidores
      \begin{itemize}
            \item 6.1. Migração e transformação de servidores para provedores de nuvem;
            \item 6.2. Desafios na área de descoberta do ambiente fonte;
            \item 6.3. Definição de ambiente destino;
            \item 6.4. Decisões de estratégias de migração e transformação.
      \end{itemize}
\end{itemize}
\begin{center}
Procedimentos Metodológicos
\end{center}
Aulas expositivas; estudos dirigidos; seminários; vídeos; dinâmicas de grupo; visitas técnicas; palestras.
\begin{center}
Recursos Didáticos
\end{center}
Quadro branco e pincel; computador; internet; projetor de multimídia.
\begin{center}
Avaliação
\end{center}
Trabalho escrito; apresentação de seminários; relatórios; avaliação escrita.
\begin{center}
Bibliografia Básica
\end{center}
\begin{itemize}
      \item 1. ERL, Thomas. Cloud Computing: Concepts, Technology \& Architecture. Editora Prentice Hall. 2013.
      \item 2. VELTE, Anthony T. Cloud Computing. Computação Em Nuvem: Uma Abordagem Prática. Alta Books. 2012.
      \item 3. FOX, Armando; PATTERSON, David. Construindo Software como Servico (SaaS): Uma Abordagem Agil Usando Computacao em Nuvem (Portuguese Edition). Editora Strawberry Canyon LLC. 2015.
\end{itemize}
\begin{center}
Bibliografia Complementar
\end{center}
\begin{itemize}
      \item 1. BRIAN, J.S. Chee; FRANKLIN, Jr., Curtis. Computação em Nuvem: Cloud Computing - Tecnologias e Estratégias. 1. ed. M.Books. 2013.
      \item 2. Above the Clouds: A Berkeley View of Cloud Computing. Relatório Técnico. 2009.
      \item 3. BIRMAN, Kenneth. Guide to Reliable Distributed Systems: Building High-Assurance Applications and Cloud-Hosted Services. Springer. 2012.
      \item 4. VERAS, Manoel. Computação em Nuvem: Nova Arquitetura de TI. 1. ed. 2015.
      \item 5. KAVIS, Michael J. Architecting the Cloud: Design Decisions for Cloud Computing Service Models (SaaS, PaaS, and IaaS). Editora Wiley. 2014.
\end{itemize}
\begin{center}
Software(s) de Apoio:
\end{center}
\begin{itemize}
      \item IDEs.
\end{itemize}

\subsection{Histórico e Contextualização dos sistemas de computação em Nuvem.}
...

\section{Aula - 10/09/2025}
Conteúdos ministrados: Aula 02: - Surgimento dos Containers -Orquestração de Containers -Kubernetes

\subsection{Aula 02: - Surgimento dos Containers -Orquestração de Containers -Kubernetes}
...

\section{Aula - 16/09/2025}
Conteúdos ministrados: Introdução aos tipos de nuvens e Benefícios, desafios e riscos das plataformas de serviços.

\subsection{Introdução aos tipos de nuvens e Benefícios, desafios e riscos das plataformas de serviços.}
...

\section{Aula - 17/09/2025}
Conteúdos ministrados: SAAS, IAAS e PAAS - Introdução a Containers

\subsection{SAAS, IAAS e PAAS - Introdução a Containers}
...

\section{Aula - 23/09/2025}
Conteúdos ministrados: Operações com container;

\subsection{Operações com container;}
...

\section{Aula - 24/09/2025}
Conteúdos ministrados: Operações com container.

\subsection{Operações com container.}
...

\section{Aula - 30/09/2025}
Conteúdos ministrados: Configurações de limites no container docker.

\subsection{Configurações de limites no container docker.}
...

\section{Aula - 07/10/2025}
Conteúdos ministrados: Configuração de recursos de hardware para conteineres.

\subsection{Configuração de recursos de hardware para conteineres.}
...

\section{Aula - 08/10/2025}
Conteúdos ministrados: Volumes em docker

\subsection{Volumes em docker}
...

\section{Aula - 14/10/2025}
Conteúdos ministrados: Configuração de volume reutilizável com read-only em docker.

\subsection{Configuração de volume reutilizável com read-only em docker.}
...

\section{Exam - 21/10/2025}
Content of the exam: docker concepts and fundamentals.

\subsection{Question 6}
We were presented a broken dockerfile. To create the fixed one, the following command was used:

\begin{minted}{bash}
      touch Dockerfile && echo "FROM nginx:latest
      COPY ./sites /usr/share/nginx/html
      EXPOSE 80" >> Dockerfile
\end{minted}

Before building the image and running the container, we need to create the folder "sites" and a file to change the \textbf{nginx} default \textbf{index.html}. This was done:

\begin{minted}{bash}
      mkdir sites
      touch sites/index.html
      echo "testando" >> sites/index.html
\end{minted}

To build the image:
\begin{minted}{bash}
      docker build -t exam:latest .
\end{minted}

To run a container:
\begin{minted}{bash}
      # You can also assign a name to it using --name
      docker run -ti -d -p 8080:80 exam:latest
\end{minted}

To get the default page and check it's contents:
\begin{minted}{bash}
      wget localhost:8080
      cat index.html 
\end{minted} 

\section{Lecture - 22/10/2025}

\bibliographystyle{alpha}
\bibliography{sample}

\end{document}
